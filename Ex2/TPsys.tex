\documentclass{article}

\pagestyle{empty}
%\pagestyle{myheadings}
%\pagenumbering

\input ../../../../amiga
%\markright{D�partement Mesures Physiques	\hfill  IUT La Creusot}
%\markright{D�partement Mesures Physiques	\hfill IUT Le Creusot - }

%\thepage
%Programmation en Java - 1�re ann�e

\parindent 10mm
\oddsidemargin 0mm
\topmargin -25mm
\textwidth 160mm
\textheight 255mm

\long\def\comment#1{}
\newcounter{quest}
\newtheorem{theorem}{Theorem}
\newtheorem{exercice}{$\bullet$ Exercice}
\newtheorem{exercise}{$\bullet$ Exercice}
\newtheorem{syntaxe}{$\bullet$ Syntaxe}
\newtheorem{remarque}{\indent Remarque}
\newcommand{\bv}{\begin{verbatim}}
\newcommand{\ev}{\end{verbatim}}

%\newtheorem{question}{\rm{\rm Q.\usecounter{quest}}}
%\newcommand{\q}{Q. \protect{\usecounter{quest}}}
%\addtocounter{quest}{1}

%\newtheorem{question}{}[exercice]%h�rite du compteur exercice
\newtheorem{question}{}[exercise]%h�rite du compteur exercice

\begin{document}

%\ \vspace{-35mm}

\begin{center}
	{
	Master {\sl ViBot}	\hfill  Universit� de Bourgogne\\
	Digital Signal Processing \hfill Le Creusot\\ \  \\
	} Practical works -- n$^o$2 \\ \ \\
	{\sl
	Systems\\
	 \ \\ %\ \\
	}
\end{center}

\ \\ 

{\exercise -- Causality} \ %\\ 

{\question\ } Considering the system defined by the equation $y_k=(x_k+x_{k+1})/2$, check its causality property by examining the response to the signal $H(k-4)$ or {\tt step(4,N)}. When plotting, include the abscissa range $[1:N]$.
{\question\ } Propose a modification to obtain a causal version



\ \\ %\\

{\exercise -- Stability} \ %\\ 

{\question\ } Program the primitive (accumulator) operator {\tt prim(f)} applied on the signal {\tt f} of length {\tt N}. The value of the vector returned by {\tt prim} at the index {\tt k} will correspond to ${\tt F}_{\tt k}$ with {\tt k} $\leq$ {\tt N}. Note $F_k=\sum_{q=-\infty}^{k}f_k$. Discuss on the result of the primitive operator applied to the signal $H(k-4)$. Is the primitive operator stable~?

{\question\ } What is the impulse response of the primitive operator (in the discrete domain)~?

{\question\ } Test the stability of the system defined by the equation: $y_k=x_k+2y_{k-1}$. Plot the impulse response.
{\question\ } Test the stability of the system defined by the equation: $y_k=x_k+y_{k-1}/3$. Plot the impulse response. Write the response $y$ as a convolution operation (truncate the impulse response).

\ \\ %\\

{\exercise -- Invariance and linearity} \ %\\ 

{\question\ } Define the following signals: ${\tt x_a=[0\, 0\,  0 \, 0 \, 1 \, 2 \, 3 \, 4 \, 5 \, 0 \, 0 \, 0 \, 0\,  0\,  0\,  0\,  0\,  0\,  0]; x_b=[0\,  0 \, 0 \, 0 \, 0 \, 0 \, 0 \, 0 \, 0 \, 4 \, 3 \, 2 \, 1 \, 0 \, 0 \, 0 \, 0 \, 0 \, 0];}$. Compute the responses $y_a$, $y_b$ according to the equation $y=3x_{k-1}-2x_k+x_{k+1}$
{\question\ } Prove the system defined by the previous equation is linear (and invariant). Write the equation as a convolution equation.
{\question\ } Propose a nonlinear/noninvariant system.

\ \\ %\\

{\exercise -- Convolution} 

{\question\ } Try to write a simple version of the convolution function (do no process the limits of the signal).

{\question\ } Generate ({\tt matlab} function {\tt randn}) an observation $x_n$ (length 1000 points or more) of the normal/gaussian random process ${\cal N}$. Plot the distribution of the values of this observation. Compute the convolution product ($y$) of $x_n$ with the values $h=[18\,\, 8\,\, 5\,\, 2\,\, 1]$ ({\tt conv(x, h, 'same')}). Compute the cross-correlation of $x_n$ with $y$ and observe the result. Conclusion~?  


\end{document}
















